\documentclass[a4paper]{article}
\usepackage{listings}

\newcommand*{\code}[1]{\lstinline{#1}}

\begin{document}

\section{Components}

The program consists of the following components:

\paragraph*{Scanner:} The scanner has one public method: \code{getToken}, 
which scans and returns the next token from the program. The 
constructor takes references to the program string and an 
\code{ErrorHandler}, to which it passes any lexical errors it finds.
A \code{Token} is a data structure that contains the a \code{value} and 
a \code{span}: which positions of the code it covers.
The value of a token can be an integer/string literal, 
identifier, keyword, operator or punctuation.

\paragraph*{Parser:}

\paragraph*{ErrorHandler:}

\paragraph*{Visitors:}

\section{Token patterns}

\begin{itemize}
  \item Identifiers and reserved keywords
  \item String literals
  \item Integer constants
  \item Operators
  \item Comments
  \item Other
\end{itemize}

\section{Parsing}

\section{Abstract Syntax Trees}

\section{Error handling}

\section{Testing}

\end{document}