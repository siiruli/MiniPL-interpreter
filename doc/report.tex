\documentclass[a4paper]{article}
\usepackage{listings}
\usepackage{amsmath}
\newcommand*{\code}[1]{\texttt{#1}}
\usepackage{parskip}

\begin{document}

\section{Architecture}

The codebase has the following main components:

\begin{itemize}
  \item \code{MiniPL}: The overall interpreter. Has methods 
        \code{runFile(filename)} and \code{run(program)} for 
        interpreting a given program.
  \item \code{Scanner}:  
        Scans the input program and produces tokens
  \item \code{Parser}:
        Parses the tokens and produces an Abstract Syntax Tree (AST).
  \item \code{SemanticAnalyzer}: Finds semantic
        errors, excluding type errors.
  \item \code{TypeChecker}: Finds type errors.
  \item \code{InterpreterVisitor}: Executes the program.
  \item \code{ErrorHandler}: Keeps track of errors found, prints them if asked.
\end{itemize}

Most of the data is stored in small structs or variants, described 
below:
\begin{itemize}
  \item \code{Position}: A position in the program code (row and column).
  \item \code{Span}: The start and end positions of a token, node or other object.
  \item \code{Token}: One token. Contains a \code{Span} and 
  a \code{TokenValue}.
  \item \code{AstNode}: A variant that holds an AST-node 
        (see Section \ref{sect:AST}).
\end{itemize}

\clearpage
\section{Scanning}

The \code{Scanner} is an ad-hoc scanner with one-character look-ahead.
It has one public method: \code{getToken()}, which scans the next token 
in the program.
It iterates through the program using a small helper-class 
\code{ProgramIterator}, which has methods \code{currentChar()}, 
\code{peekChar()}, and \code{move()} for iterating. The 
iterator additionaly keeps track of the current program \code{Position}, 
which the scanner uses to add a \code{Span} to the tokens.

The scanner always returns a viable token. After reaching the 
end of the program, it just keeps returning an \code{Eof} token 
with the same span. When encountering an error, it skips the 
wrong character and returns the current token or scans the next token, 
depending on where the error happened. For information about 
error messages, see Section \ref{sect:errors}.


\subsection{Token patterns}

There are four five of tokens: identifiers, Keywords, 
literals (integers or strings), operators, and delimiters.
Below is a regular definitions of the possible tokens. 
\begin{itemize}
  \item[] \code{VarIdent} $\rightarrow$ \verb?[A-Za-z][A-Za-z0-9_]*?
  \item[] \code{Literal} $\rightarrow$ \verb_[0-9]+|"([^\n"]|\\(n|t|\n|.))*"_
  \item[] \code{Operator} $\rightarrow$ \verb_+|-|*|/|<|=|\&|!_
  \item[] \code{Delimiter} $\rightarrow$ \verb_:=|;|:|..|(|)|$_
  \item[] \code{TokenValue} $\rightarrow$ \verb_VarIdent|Literal|Operator|Delimiter_
\end{itemize}
The \code{Keyword} tokens form an exception, as they are not scanned 
based on a regular definition. Whenever an identifier is scanned, 
the scanner checks whether it is actually a keyword, and returns 
a \code{Keyword} if necessary.
The possible keywords are  "var", "for", "end", "in", "do", "read", "print", 
"int", "string", "bool", "assert", "if", and "else".

The names above correspond to the type names in the code.  
In the code, identifiers are stored as strings, and literals in a 
struct of type \code{Literal}, containing a value which is 
an integer/string variant, while
values of type \code{Operator}, \code{Delimiter}, and \code{Keyword} 
are stored as enum classes.
The values are stored in the \code{std::variant}
\code{TokenValue}, which may hold any of the above types. 

In addition to the tokens, two types of comments are allowed in 
the language. Single-line comments have the format \verb_//.*_, 
and end at the end of the line. Multi-line comments have the format
\verb_/*...*/_, and can be nested. Nested comments do not allow 
a regular definition, so they form yet another exception. The 
scanner skips all comments and whitespace between tokens.



\clearpage
\section{Parsing}

The parser is a recursive-descent parser with one-token look-ahead.
It iterates through the tokens with a small helper class 
\code{TokenIterator}, which provides the methods \code{currentToken()} 
and \code{nextToken()} (which moves to the next token and returns that).

\subsection{LL(1) grammar}


\newcommand{\cfgvar}[1]{$<$#1$>$}
\newcommand{\cfgrule}[2]{\text{\cfgvar{#1}} &\rightarrow \text{#2}}
\newcommand{\cfgterm}[1]{\textbf{#1}}
The terminals are \cfgvar{literal}, \cfgvar{op}, \cfgvar{ident}, 
\cfgvar{unary$\_$op}, the end-of-file symbol \$\$, punctuation 
and the bolded Keywords.
\begin{align*}
  \cfgrule{prog}{\cfgvar{stmts} \$\$} \\
  \cfgrule{stmts}{\cfgvar{stmt}; \cfgvar{stmts} $\mid$ $\varepsilon$} \\
  \cfgrule{stmt}{\cfgvar{decl} $\mid$ 
                 \cfgvar{assign} $\mid$
                 \cfgvar{for} $\mid$ 
                 \cfgvar{read} $\mid$ 
                 \cfgvar{print} $\mid$ 
                 \cfgvar{if} } \\
  \cfgrule{decl}{
    \cfgterm{var} \cfgvar{ident} : \cfgvar{type} \cfgvar{delc\_assign}
  } \\
  \cfgrule{decl\_assign}{
     := \cfgvar{expr} $\mid$ $\varepsilon$
  }\\
  \cfgrule{assign}{
    \cfgvar{ident} := \cfgvar{expr}
  } \\
  \cfgrule{for}{
    \cfgterm{for} \cfgvar{ident} \cfgterm{in} 
    \cfgvar{expr}..\cfgvar{expr} \cfgterm{do} \cfgvar{stmts} 
    \cfgterm{end for}
  } \\
  \cfgrule{read}{
    \cfgterm{read} \cfgvar{ident}
  } \\
  \cfgrule{print}{
    \cfgterm{print} \cfgvar{expr}
  } \\ 
  \cfgrule{if}{
    \cfgterm{if} \cfgvar{expr} \cfgterm{do} \cfgvar{stmts}
    \cfgvar{else} \cfgterm{end if}
  } \\
  \cfgrule{else}{
    \cfgterm{else} \cfgvar{stmts} $\mid$ $\varepsilon$
  } \\
  \cfgrule{expr}{ 
    \cfgvar{opnd} \cfgvar{expr$\_$tail} 
    $\mid$ \cfgvar{unary$\_$op} \cfgvar{opnd}
  } \\
  \cfgrule{expr$\_$tail}{
    \cfgvar{op} \cfgvar{opnd} $\mid$ $\varepsilon$
  } \\
  \cfgrule{opnd}{
    \cfgvar{literal} $\mid$ \cfgvar{ident} 
    $\mid$ \cfgterm{(} \cfgvar{expr} \cfgterm{)}
  } \\
\end{align*}

\clearpage
\label{sect:AST}
\section{Abstract Syntax Trees}

All AST-nodes inherit the \code{AstNodeBase} class, which contains 
a \code{Span} for error messages. The type \code{AstNode} is a 
\code{std::variant} which may contain an AST-node of any of the 
following types:

\begin{itemize}
  \item \code{DeclAstNode}: a declaration.
  \begin{itemize}
    \item[] \code{VarIdent varId}
    \item[] \code{Type type} 
    \item[] \code{optional<ExprAstNode> expr}
  \end{itemize}
  \item \code{AssignAstNode}: assignment
  \begin{itemize}
    \item[] \code{VarIdent varId}
    \item[] \code{ExprAstNode expr}
  \end{itemize}
  \item \code{ForAstNode}: for-statement
  \begin{itemize}
    \item[] \code{VarIdent varId}: loop variable
    \item[] \code{ExprAstNode startExpr, endExpr}: start and end of the range
    \item[] \code{StatementsAstNode stmts}: the statements inside the loop
  \end{itemize}
  \item \code{IfAstNode}: if-statement
  \begin{itemize}
    \item[] \code{ExprAstNode expr}
    \item[] \code{StatementsAstNode ifStatements}
    \item[] \code{StatementsAstNode elseStatements}
  \end{itemize}
  \item \code{ReadAstNode}: read-statement
  \begin{itemize}
    \item[] \code{VarIdent varId}
  \end{itemize}
  \item \code{PrintAstNode}: print-statement
  \begin{itemize}
    \item[] \code{ExprAstNode expr}
  \end{itemize}
  \item \code{StatementsAstNode}: a list of statements
  \begin{itemize}
    \item[] \code{vector<AstNode> stmts}: list of statements
  \end{itemize}
  \item \code{ExprAstNode}: expression
  \begin{itemize}
    \item[] \code{Operator op}: enum storing the operator
    \item[] \code{OpndAstNode *opnd1}
    \item[] \code{OpndAstNode *opnd2}: null for unary operands.
  \end{itemize}
  \item \code{OpndAstNode}: operand
  \begin{itemize}
    \item[] \code{Operand opnd}, which is one of 
    \code{Literal, VarIdent, ExprAstNode}
  \end{itemize}
\end{itemize}

\section{Semantic analysis}

\label{sect:errors}
\section{Error handling}

All components are given a reference to the \code{ErrorHandler}, 
which they use to raise errors when something unexpected happens.
The \code{ErrorHandler} prints out error messages and keeps track 
of whether errors have been encountered. 
The components then do something component-specific to recover and 
continue processing the input program. If the \code{ErrorHandler} 
has errors after the scanning and parsing passes, the interpreter 
ends the process. Otherwise it continues with semantic analysis 
and, if no errors are encountered, finally runs the program.

\subsection{Error types}

Errors are stored as structs, that all inherit from the 
base struct \code{ErrorBase}. The base struct has members 
\code{context}, \code{contextScope}, and \code{scope}, 
and a public method \code{description()}, which uses 
virtual functions of the derived classes to construct an 
error message. The derived classes and the errors they 
can represent are listed below:
\begin{itemize}
  \item \code{ScanningError}: unexpected character, newline or Eof
  \item \code{ParsingError}: unexpected token
  \item \code{SemanticError}: one of the following
  \begin{itemize}
    \item variable not declared
    \item redeclaration
    \item declaration in inner scope
    \item assignment to a constant variable
  \end{itemize}
  \item \code{TypeError}: wrong type
  \item \code{RuntimeError}: division by zero or IO failure (could not read input)
\end{itemize}

\subsection{Error messages}

\subsection{Error recovery}

The \code{Scanner} always returns a token when asked, whether or 
not it runs into errors while processing. When a problem is 
encountered and the current token can not be scanned, the scanner 
skips it and scans the next token (after raising an error to the 
handler). In some cases, e.g. when encountering a newline while 
scanning a string, the \code{Scanner} can return the the token it 
was scanning while the error happened. When the end of the 
program has been reached, the Scanner returns an end-of-file token.

The \code{Parser} has an exception-based racovery approach. When the 
\code{match}-function can not match the current token, an error 
is raised and an exception thrown. The exception is caught 
in the \code{statements} function, and tokens are skipped until
the current token is a semicolon or one of the Keywords 
\textbf{var}, \textbf{for}, \textbf{read}, \textbf{print}, 
or \textbf{if}. The parser uses these to find the start of a 
new statements and resumes parsing from that point.


The \code{TypeChecker} can return a \code{Broken} type from e.g.
a malformed expression or an undeclared variable. Type checks 
for broken types are automatically skipped to avoid cascading 
errors. Thus a long expression with a type error in inner 
parenthesis only produces one error. In case of multiple 
declarations of the same variable, the first declaration is 
assumed to be correct.

When \code{InterpreterVisitor} produces a runtime error, it 
raises an error to the \code{ErrorHandler} and throws an exception. 
The exception is caught by \code{MiniPL} which asks the 
\code{ErrorHandler} to print the error message and stops running the 
program.

\section{Testing}

Automated testing of the interpreter was implemented with the 
googletest framework. All tests and sample programs were written 
by hand, based on whatever 
things I thought were important. When encountering a bug, I would 
fix the bug and write a test for it. 


The scanner is well tested with unit tests 
covering all different token types, while other components have 
only a few limited unit tests. The whole system is tested with 
integration tests that check whether running the test programs 
produces the expected behaviour. Additionally, there is one 
test that runs all programs in the \code{samples} directory 
and (hopefully) checks that the interpreter doesn't crash.

There are no automatic tests for the error handling and recovery 
functionalities. These were tested by hand, by running the incorrect 
sample programs and looking at the output.


\end{document}